% JUNYOUNG PARK RESUME TEX
% Author: Junyoung Park (mail@jyp.me)
% History:
%   2015/07/21 - first version in KAIST
%   2019/01/26 - second major update in UXFactory, Inc.

%---------------------------------------------------------------------------
% PACKAGES AND OTHER DOCUMENT CONFIGURATIONS
%---------------------------------------------------------------------------
\documentclass{resume} % Use the custom resume.cls style

\usepackage[left=0.6in,top=0.5in,right=0.6in,bottom=0.5in]{geometry} % Document margins

% to use the fontspec package, compile with luluatex only
\usepackage{fontspec}
\setmainfont{Times New Roman}

\begin{document}

% top information
\name{Junyoung~Park}
\job{AI System-on-Chip Engineer \& Director}

\contactinfo {
  \location{Seongnam-si 13105, Gyeonggi-do, Republic of Korea}
  \phone{+82-10-2542-5500}
  \email{junyoung@uxf.ai}
  \homepage{http://www.junyoungpark.com}
  \linkedin{https://www.linkedin.com/in/parkjunyoung}
}

%% Make the header extend all the way to the right, if you want.
\makecvheader

%---------------------------------------------------------------------------
% WORK SECTION
%---------------------------------------------------------------------------

\begin{section}{Work Experience}

\begin{subsection}{UX Factory, Inc.}{Aug. 2015 - Present}{Co-founder and Chief Executive Officier}{}
\item Created a company that delivers the world's leading AI solutions derived from SW-SoC technology
\item Participated in 6 government R\&D projects for AI \& SoC with major Korean fabless companies
\end{subsection}

\begin{subsection}{KAIST}{Jan. 2015 - Aug. 2015}{Postdoctoral Fellow}{}
\item SoC architecture exploration for Vision \& Deep Learning in the hardware-software codesign methodology
\end{subsection}

\begin{subsection}{Samsung Mobile Processor Innovation Lab}{Sep. 2014 - Dec. 2014}{Research Intern}{}
\item Established a System-C TLM based simulator for early top-level system exploration  in many-core SoCs
\end{subsection}

\end{section}

%---------------------------------------------------------------------------
% EDUCATION SECTION
%---------------------------------------------------------------------------

\begin{section}{Education}

\begin{subsection}{Ph.D. in Electrical Engineering, KAIST}{Aug. 2014}{}{}%{National Full-Scholarship}{Daejeon, Korea}
\item Thesis: Energy-efficient Context-aware Real-Time Object Recognition Processor
%\item Kim Choong-Ki Scholarship Award for Outstanding Research Accomplishment
%\item Intel/Analog Devices/Catalyst Foundation CICC Student Scholarship Award
\item Designed and implemented an energy-efficient vision SoC for context-aware object recognition \\
- presented and demonstrated at \emph{IEEE International Solid-State Circuits Conference}
\end{subsection}

\begin{subsection}{M.S. in Electrical Engineering, KAIST}{Feb. 2011}{}{}%{National Full-Scholarship}{Daejeon, Korea}
\item Thesis: On-chip Learning Multi-class Support Vector Machine Processor
%\item Designed and implemented a stereo matching processor to accelerate belief propagation algorithm
%\item Korean Science \& Technology Research Scholarship Award
%\item Eun Jong-Kwan Scholarship Award for Honor of First Place M.S. Freshman
\item Designed and implemented a traffic sign recognition SoC for advanced driver assistance system \\
- published in \emph{IEEE Journal of Solid-State Circuits}
\end{subsection}

\begin{subsection}{B.S. in Electrical Engineering, KAIST}{Feb. 2009}{}{}
\item Graduated with \emph{Summa Cum Laude}
\end{subsection}

\end{section}

%---------------------------------------------------------------------------
% AWARDS AND ACTIVITY SECTION
%---------------------------------------------------------------------------
\begin{section}{Awards and Invited Talks}

\begin{subsection}{Awards}{}{}{}
\item Kim Choong-Ki Scholarship Award for Outstanding Research Accomplishments \hfill Apr. 2013
\item IEEE International Solid-State Circuits Conference Academic Demo Session \hfill Feb. 2013
\item Intel/Analog Devices/Catalyst Foundation CICC Student Scholarship Award \hfill Sep. 2012
\item Eun Jong-Kwan Scholarship Award for Honor of First Place M.S. Freshman \hfill Apr. 2009
\item Korean Science \& Technology Research Scholarship Award \hfill Feb. 2009 - Feb. 2011
\end{subsection}

\begin{subsection}{Invited Talks}{}{}{}

\item Bringing Deep Learning to the Edge,
{\small\textit{Korea Institute of Science and Technology}, Dec., 2019.}

\item Technical Directions for the Next-generation AI,
{\small\textit{Korea Educational Center of Future Technology}, Mar., 2019.}

\item Deep learning SW framework and ASIC for AI SoC,
{\small\textit{Electronics and Telecommunications Research Institute.}, Sept., 2018.}

\item Embedded Deep Neural Network SoC,
{\small\textit{Korea Electronics Technology Institute}, Sept., 2018.}

\item Embedded Deep Neural Network SoC,
{\small\textit{Electronics and Telecommunications Research Institute}, Mar., 2017.}

\item Embedded Deep Neural Network SoC:  deep learning to mobile devices,
{\small\textit{Deep Neural Network SoC Workshop}, Aug., 2016.}

\item Low-power Pattern recognition SoC with bio-inspired architecture for intelligent cognitive service,
{\small\textit{IEEE International Conference on Intelligent Robots and Systems (IROS) Workshops}, Sept., 2015.}

\item An energy-efficient SoC for real-time context-aware object recognition,
{\small\textit{Samsung Research America}, Sept., 2014.}

\item An energy-efficient heterogeneous many-core processor for real-world object recognition,
{\small\textit{Qualcomm}, Feb., 2013.}

\end{subsection}

\end{section}



%---------------------------------------------------------------------------
% PUBLICATION SECTION
%---------------------------------------------------------------------------
\begin{section}{Publications}

\begin{subsection}{Journal Papers}{}{}{}

\item An Energy-Efficient Embedded Deep Neural Network Processor for High Speed Visual Attention in Mobile Vision Recognition SoC,
\emph{IEEE Journal of Solid-State Circuits, vol.PP, no.99, pp.1-9}, July. 2016. \\
{\small S. Park, I. Hong, \underline{\bf Junyoung Park}, and H.-J. Yoo.}

\item A 0.5 V 54 μW Ultra-Low-Power Object Matching Processor for Micro Air Vehicle Navigation,
\emph{IEEE Transactions on Circuits and Systems I: Regular Papers}, vol.63, no.3, pp.359-369, Mar. 2016. \\
{\small Y. Kim, I. Hong, \underline{\bf Junyoung Park}, H.-J. Yoo.}

\item An Energy-efficient and Scalable Deep Learning/Inference Processor With Tetra-Parallel MIMD Architecture for Big Data Applications,
\emph{IEEE Transactions on Biomedical Circuits and Systems}, vol.9, no.6, pp.838-848, Dec. 2015. \\
{\small S. Park, \underline{\bf Junyoung Park}, K. Bong, D. Shin, J. Lee, S. Choi, H.-J. Yoo.}

\item A Vocabulary Forest Object Matching Processor With 2.07 M-Vector/s Throughput and 13.3 nJ/Vector Per-Vector Energy for Full-HD 60 fps Video Object Recognition, \emph{IEEE Journal of Solid-State Circuits}, vol.50, no.4, pp.1059-1069, Apr. 2015. \\
{\small K.J. Lee, G. Kim, \underline{\bf Junyoung Park}, and H.-J. Yoo.}

\item Intelligent Network-on-Chip With Online Reinforcement Learning for Portable HD Object Recognition Processor, \emph{IEEE Transactions on Circuits and Systems I: Regular Papers}, vol.61, no.2, pp.476-484, Feb. 2014. \\
{\small \underline{\bf Junyoung Park}, I. Hong, G. Kim, B.-G. Nam, and H.-J. Yoo.}

\item A 320 mW 342 GOPS Real-Time Dynamic Object Recognition Processor for HD 720p Video Streams, \emph{IEEE Journal of Solid-State Circuits}, vol.48, no.1, pp.33-45, Jan. 2013. \\
{\small J. Oh, G. Kim, \underline{\bf Junyoung Park}, I. Hong, S. Lee, J.-Y. Kim, J.-H. Woo, H.-J. Yoo.}

\item Low-Power, Real-Time Object-Recognition Processors for Mobile Vision Systems \emph{IEEE Micro}, vol.32, no.6, pp.38-50, Nov.-Dec. 2012. \\
{\small J. Oh, G. Kim, I. Hong, \underline{\bf Junyoung Park}, S. Lee, J.-Y. Kim, J.-H. Woo, H.-J. Yoo.}

\item A 92-mW Real-Time Traffic Sign Recognition System With Robust Illumination Adaptation and Support Vector Machine, \emph{IEEE Journal of Solid-State Circuits}, vol.47, no.11, pp.2711-2723, Nov. 2012. \\
{\small \underline{\bf Junyoung Park}, J. Kwon, J. Oh, S. Lee, J.-Y. Kim, and H.-J. Yoo.}

\item A 345 mW Heterogeneous Many-Core Processor With an Intelligent Inference Engine for Robust Object Recognition \emph{IEEE Journal of Solid-State Circuits}, vol.46, no.1, pp.42-51, Jan. 2011. \\
{\small S. Lee, J. Oh, \underline{\bf Junyoung Park}, J. Kwon, M. Kim, H.-J. Yoo.}

\item A 118.4 GB/s Multi-Casting Network-on-Chip With Hierarchical Star-Ring Combined Topology for Real-Time Object Recognition, \emph{IEEE Journal of Solid-State Circuits}, vol.45, no.7, pp.1399-1409, July 2010. \\
{\small J.-Y. Kim, \underline{\bf Junyoung Park}, S. Lee, M. Kim, J. Oh, and H.-J. Yoo.}

\end{subsection}

\begin{subsection}{Conference Papers (First Authored Only - 25 Papers in Total)}{}{}{}

\item A High-throughput 16x Super Resolution Processor for Real-Time Object Recognition SoC, \emph{IEEE European Solid-State Circuits Conference}, pp.259-262, 16-20 Sep. 2013. \\
{\small \underline{\bf Junyoung Park}, B.-G. Nam, H.-J. Yoo.}

\item A multi-granularity parallelism object recognition processor with content-aware fine-grained task scheduling, \emph{IEEE Symposium on Low-Power and High-Speed Chips}, pp.1-3, 17-19 April 2013. \\
{\small \underline{\bf Junyoung Park}, I. Hong, G. Kim, Y. Kim, K. Lee, S. Park, K. Bong, H.-J. Yoo.}

\item A 646 GOPS/W Multi-classifier Many-core Processor with Cortex-like Architecture for Super-Resolution Recognition, \emph{IEEE International Solid-State Circuits Conference}, Feb., 2013. \\
{\small \underline{\bf Junyoung Park}, I. Hong, G. Kim, Y. Kim, K. Lee, S. Park, K. Bong, and H.-J. Yoo.}

\item Online Reinforcement Learning NoC for Portable HD Object Recognition Processor, \emph{IEEE Custom Integrated Circuits Conference}, Sep., 2012. \\
{\small \underline{\bf Junyoung Park}, I. Hong, G. Kim, J. Oh, S. Lee, H.-J. Yoo.}

\item A 92mW Real-Time Traffic Sign Recognition System with Robust Light and Dark Adaptation, \emph{IEEE Asian Solid-state Circuit Conference}, Nov., 2011. \\
{\small \underline{\bf Junyoung Park}, J. Kwon, J. Oh, S. Lee, H.-J. Yoo.}

\item A 30fps Stereo Matching Processor Based on Belief Propagation with Disparity-Parallel PE Array Architecture, \emph{IEEE International Symposium on Circuits and Systems}, Mar., 2010. \\
{\small \underline{\bf Junyoung Park}, S. Lee, H.-J. Yoo.}

\end{subsection}

\begin{subsection}{Patents}{}{}{}

\item {\small KR101638095 METHOD FOR PROVIDING USER INTERFACE THROUGH HEAD MOUNT DISPLAY BY USING GAZE RECOGNITION AND BIO-SIGNAL, AND DEVICE, AND COMPUTER-READABLE RECORDING MEDIA USING THE SAME}

\item {\small KR101907028 Analog Digital Interfaced SRAM Structure}

\item {\small KR101841744 Stereo Image Matching System integrated CMOS Image Sensor and Method thereof}

\item {\small KR1020170184725 SRAM Structure for Supporting Transposed Read}

\item {\small KR1020180009756 Low Power Face Recognition System using CMOS Image Sensor Integrated with a Face Detector}

\item {\small KR101190000 SUPPORT VECTOR MACHINE PROCESSOR AND MEMORY MANAGEMENT METHOD THEREOF}

\end{subsection}

\end{section}



%---------------------------------------------------------------------------
%	TECHNICAL STRENGTHS SECTION
%---------------------------------------------------------------------------

% \begin{section}{Skills}

% \begin{tabular}{ @{} >{\bfseries}l @{\hspace{6ex}} l }
% Computer Languages & Verilog HDL(incl. SystemVerilog), C/C++, JAVA, Python, Perl\\
% EDA Tools & Front-to-back full chip implementation (RTL/schematic to P\&R) \\
% \end{tabular}

% \end{section}

% \begin{section}{Languages}
% Native Korean \& Fluent English
% \end{section}

% \begin{section}{References}
% Available upon Request
% \end{section}

%----------------------------------------------------------------------------------------
%	EXAMPLE SECTION
%----------------------------------------------------------------------------------------

%\begin{rsection}{section Name}

%section content\ldots

%\end{rSection}

%----------------------------------------------------------------------------------------

\end{document}
