% Resume LaTex
% Author: Junyoung Park (junyoung.park@kaist.ac.kr)
% Date: 2015/07/21

%---------------------------------------------------------------------------
% PACKAGES AND OTHER DOCUMENT CONFIGURATIONS
%---------------------------------------------------------------------------
\documentclass{resume} % Use the custom resume.cls style

\usepackage[left=0.75in,top=0.6in,right=0.75in,bottom=0.6in]{geometry} % Document margins

\name{Junyoung~Park}
\address{Semiconductor System Lab}
\address{\#1233 E3-2 (Dept. of Electrical Engineering), KAIST}
\address{Daejeon 305-701, Republic of Korea}
\phone{+82-10-2542-5500}
\email{junyoung.park@kaist.ac.kr}
\webpage{http://www.junyoungpark.com}
\begin{document}

\begin{section}{Objective}
To obtain a challenging position in the field of software-hardware co-design for application-specific SoC
\end{section}

%---------------------------------------------------------------------------
% MAJOR RESEARCH SECTION
%---------------------------------------------------------------------------

\begin{section}{Major Research Experience}

\begin{subsection}{System-on-Chip Design}{}{}{}
\item An intelligent Network-on-Chip SoC based on Machine Learning Approaches (TCAS-I 2014)
\item A mobile vision SoC integrating 21 heterogeneous cores for context-aware object recognition (ISSCC 2013)
\item An advanced driver assistant system SoC with two chip integration (JSSC 2012)
\item Participated in the design of 6 SoCs over 5 years
\end{subsection}

\begin{subsection}{Network-on-Chip Analysis}{}{}{}
\item A system-C TLM based cycle-accurate simulator for Network-on-Chip in many-core vision processors (JSSC 2011)
\item Top-level system network exploration \& verification in the application processor during the research internship
\end{subsection}

%\begin{subsection}{Embedded Platform Design}{}{}{}
%\item Worked on a mobile vision platform for unmanned aerial vehicle with integrated SoC, FPGA, and camera
%\item Experienced embedded system board design, including system specification, evaluation and debugging
%\end{subsection}

\end{section}

%---------------------------------------------------------------------------
% WORK SECTION
%---------------------------------------------------------------------------

\begin{section}{Work Experience}

\begin{subsection}{KAIST, Daejeon, Korea}{Jan. 2015 - Present}{Postdoctoral Researcher}{}
\item Advisor: Hoi-Jun Yoo
\item SoC architecture exploration for Vision \& Deep Learning in the hardware-software codesign methodology
\end{subsection}

\begin{subsection}{Samsung Mobile Processor Innovation Lab, Dallas TX, US}{Sep. 2014 - Dec. 2014}{Research Internship}{}
\item Manager: Seok-Jun Lee
\item Established a System-C TLM based simulator for early top-level system exploration  in many-core SoCs
\end{subsection}

\end{section}

%---------------------------------------------------------------------------
% EDUCATION SECTION
%---------------------------------------------------------------------------

\begin{section}{Education}

\begin{subsection}{Ph.D. in Electrical Engineering, KAIST}{Aug. 2014}{}{}%{National Full-Scholarship}{Daejeon, Korea}
\item Advisor: Hoi-Jun Yoo
\item Thesis: Energy-efficient Context-aware Real-Time Object Recognition Processor
%\item Kim Choong-Ki Scholarship Award for Outstanding Research Accomplishment
%\item Intel/Analog Devices/Catalyst Foundation CICC Student Scholarship Award
\item Designed and implemented an energy-efficient vision SoC for context-aware object recognition \\
- presented and demonstrated at \emph{IEEE International Solid-State Circuits Conference}
\end{subsection}

\begin{subsection}{M.S. in Electrical Engineering, KAIST}{Feb. 2011}{}{}%{National Full-Scholarship}{Daejeon, Korea}
\item Thesis: On-chip Learning Multi-class Support Vector Machine Processor
%\item Designed and implemented a stereo matching processor to accelerate belief propagation algorithm
%\item Korean Science \& Technology Research Scholarship Award
%\item Eun Jong-Kwan Scholarship Award for Honor of First Place M.S. Freshman
\item Designed and implemented a traffic sign recognition SoC for advanced driver assistance system \\
- published in \emph{IEEE Journal of Solid-State Circuits}
\end{subsection}

\begin{subsection}{B.S. in Electrical Engineering, KAIST}{Feb. 2009}{}{}
\item Graduated with \emph{Summa Cum Laude}
\end{subsection}

\end{section}

%---------------------------------------------------------------------------
% AWARDS AND ACTIVITY SECTION
%---------------------------------------------------------------------------
\begin{section}{Awards and Activities}

\begin{subsection}{Awards}{}{}{}
\item Kim Choong-Ki Scholarship Award for Outstanding Research Accomplishments \hfill Apr. 2013
\item IEEE International Solid-State Circuits Conference Academic Demo Session \hfill Feb. 2013
\item Intel/Analog Devices/Catalyst Foundation CICC Student Scholarship Award \hfill Sep. 2012
\item Eun Jong-Kwan Scholarship Award for Honor of First Place M.S. Freshman \hfill Apr. 2009
\item Korean Science \& Technology Research Scholarship Award \hfill Feb. 2009 - Feb. 2011
\end{subsection}

\begin{subsection}{Activities}{}{}{}
\item Teaching/Research Assistant in Electrical Engineering, KAIST \hfill Feb. 2009 - Aug. 2014
\end{subsection}

\end{section}



%---------------------------------------------------------------------------
% PUBLICATION SECTION
%---------------------------------------------------------------------------
\begin{section}{Publications}

\begin{subsection}{Journal Papers}{}{}{}

\item A Vocabulary Forest Object Matching Processor With 2.07 M-Vector/s Throughput and 13.3 nJ/Vector Per-Vector Energy for Full-HD 60 fps Video Object Recognition, \emph{IEEE Journal of Solid-State Circuits}, vol.50, no.4, pp.1059-1069, Apr. 2015. \\
K.J. Lee, G. Kim, \underline{\bf Junyoung Park}, and H.-J. Yoo.

\item Intelligent Network-on-Chip With Online Reinforcement Learning for Portable HD Object Recognition Processor, \emph{IEEE Transactions on Circuits and Systems I: Regular Papers}, vol.61, no.2, pp.476-484, Feb. 2014. \\
\underline{\bf Junyoung Park}, I. Hong, G. Kim, B.-G. Nam, and H.-J. Yoo.

\item A 320 mW 342 GOPS Real-Time Dynamic Object Recognition Processor for HD 720p Video Streams, \emph{IEEE Journal of Solid-State Circuits}, vol.48, no.1, pp.33-45, Jan. 2013. \\
J. Oh, G. Kim, \underline{\bf Junyoung Park}, I. Hong, S. Lee, J.-Y. Kim, J.-H. Woo, H.-J. Yoo.

\item Low-Power, Real-Time Object-Recognition Processors for Mobile Vision Systems \emph{IEEE Micro}, vol.32, no.6, pp.38-50, Nov.-Dec. 2012. \\
J. Oh, G. Kim, I. Hong, \underline{\bf Junyoung Park}, S. Lee, J.-Y. Kim, J.-H. Woo, H.-J. Yoo.

\item A 92-mW Real-Time Traffic Sign Recognition System With Robust Illumination Adaptation and Support Vector Machine, \emph{IEEE Journal of Solid-State Circuits}, vol.47, no.11, pp.2711-2723, Nov. 2012. \\
\underline{\bf Junyoung Park}, J. Kwon, J. Oh, S. Lee, J.-Y. Kim, and H.-J. Yoo.

\item A 345 mW Heterogeneous Many-Core Processor With an Intelligent Inference Engine for Robust Object Recognition \emph{IEEE Journal of Solid-State Circuits}, vol.46, no.1, pp.42-51, Jan. 2011. \\
S. Lee, J. Oh, \underline{\bf Junyoung Park}, J. Kwon, M. Kim, H.-J. Yoo.

\item A 118.4 GB/s Multi-Casting Network-on-Chip With Hierarchical Star-Ring Combined Topology for Real-Time Object Recognition, \emph{IEEE Journal of Solid-State Circuits}, vol.45, no.7, pp.1399-1409, July 2010. \\
J.-Y. Kim, \underline{\bf Junyoung Park}, S. Lee, M. Kim, J. Oh, and H.-J. Yoo.

\end{subsection}

\begin{subsection}{Conference Papers (First Authored Only - 25 Papers in Total)}{}{}{}

\item A High-throughput 16x Super Resolution Processor for Real-Time Object Recognition SoC, \emph{IEEE European Solid-State Circuits Conference}, pp.259-262, 16-20 Sep. 2013. \\
\underline{\bf Junyoung Park}, B.-G. Nam, H.-J. Yoo.

\item A multi-granularity parallelism object recognition processor with content-aware fine-grained task scheduling, \emph{IEEE Symposium on Low-Power and High-Speed Chips}, pp.1-3, 17-19 April 2013. \\
\underline{\bf Junyoung Park}, I. Hong, G. Kim, Y. Kim, K. Lee, S. Park, K. Bong, H.-J. Yoo.

\item A 646 GOPS/W Multi-classifier Many-core Processor with Cortex-like Architecture for Super-Resolution Recognition, \emph{IEEE International Solid-State Circuits Conference}, Feb., 2013. \\
\underline{\bf Junyoung Park}, I. Hong, G. Kim, Y. Kim, K. Lee, S. Park, K. Bong, and H.-J. Yoo.

\item Online Reinforcement Learning NoC for Portable HD Object Recognition Processor, \emph{IEEE Custom Integrated Circuits Conference}, Sep., 2012. \\
\underline{\bf Junyoung Park}, I. Hong, G. Kim, J. Oh, S. Lee, H.-J. Yoo.

\item A 92mW Real-Time Traffic Sign Recognition System with Robust Light and Dark Adaptation, \emph{IEEE Asian Solid-state Circuit Conference}, Nov., 2011. \\
\underline{\bf Junyoung Park}, J. Kwon, J. Oh, S. Lee, H.-J. Yoo.

\item A 30fps Stereo Matching Processor Based on Belief Propagation with Disparity-Parallel PE Array Architecture, \emph{IEEE International Symposium on Circuits and Systems}, Mar., 2010. \\
\underline{\bf Junyoung Park}, S. Lee, H.-J. Yoo.

\end{subsection}

\begin{subsection}{Patents}{}{}{}
\item Memory Management in Support Vector Machine Processor, Korean Patent NO. 10-1190000, 2012.
\end{subsection}

\end{section}

%---------------------------------------------------------------------------
%	TECHNICAL STRENGTHS SECTION
%---------------------------------------------------------------------------

\begin{section}{Skills}

\begin{tabular}{ @{} >{\bfseries}l @{\hspace{6ex}} l }
Computer Languages & Verilog HDL(incl. SystemVerilog), C/C++, JAVA, Python, Perl\\
EDA Tools & Front-to-back full chip implementation (RTL/schematic to P\&R) \\
\end{tabular}

\end{section}

\begin{section}{Languages}
Native Korean \& Fluent English
\end{section}

\begin{section}{References}
Available upon Request
\end{section}

%----------------------------------------------------------------------------------------
%	EXAMPLE SECTION
%----------------------------------------------------------------------------------------

%\begin{rsection}{section Name}

%section content\ldots

%\end{rSection}

%----------------------------------------------------------------------------------------

\end{document}
